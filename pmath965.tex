\documentclass[]{article}

\usepackage{amsmath}
\usepackage{amssymb}
\usepackage{amsthm}
\usepackage{tikz-cd}
\usepackage{quiver}

\newtheorem{thm}{Theorem}
\newtheorem{prop}{Proposition}
\newtheorem{lemma}{Lemma}
\theoremstyle{definition}
\newtheorem{defn}{Definition}

\newcommand{\C}{\mathbb{C}}
\newcommand{\bP}{\mathbb{P}}
\newcommand{\OO}{\mathcal{O}}
\newcommand{\sslash}{\mathbin{/\mkern-4mu/}}
\newcommand{\Spec}{\text{Spec}}
\newcommand{\m}{\mathfrak{m}}
%opening
\title{}
\author{}

\begin{document}

\section{Motivation}
Mirror symmetry is an enormous area of research. Here we provide motivation from just one perspective, which is Fano classification. Let $X$ be a smooth n-dimensional algebraic variety over $\C$. Suppose we also have a line bundle $L \to X$, with some global sections $s_1, s_2,...,s_m \in \Gamma(X,L)$. Then we can try and write down a map
\begin{align*}
	\iota: X &\rightarrow \bP^{m-1}\\
	x &\rightarrow [s_1(x):s_2(x):...:s_m(x)].
\end{align*}
This is well defined as long as there is no $x \in X$ where all the sections vanish; $s_i(x)=0$ for all $i=1,...,m$.
\begin{defn}
	A line bundle $L$ over an algebraic variety $X$ is called \emph{very ample} if there exist some global sections $s_1,...,s_m$ of $L$ for which the map $\iota$ defined above is an embedding of $X$ into $\bP^{m-1}$. \vspace{1em}
	
	If there exists a natural number $k$ such that $L^{\otimes k}$ is very ample, then we say $L$ is \emph{ample}.
\end{defn}
For example, the line bundles $\OO(n) \rightarrow \bP^{m-1}$ (not to be confused with orthogonal groups!) are very ample for all $n\geq 1$.
\begin{defn}
	The variety $X$ is \emph{Fano} if $-K_X := \bigwedge^n TX$ is \emph{ample}.	
\end{defn}
If $X$ is Fano, then it is projective, since $\bigwedge^n TX$ is very ample, meaning it has some sections which define an embedding $\iota$ of $X$ into projective space. Some examples of Fano varieties include $\bP^{n}$, any degree $d$ projective curve in $\bP^n$ with $d<n+1$, and Grassmannians. Naturally then we can ask \emph{Why study Fano varieties}? 
\begin{itemize}
	\item Fano varieties are often the ambient spaces in algebraic geometry. For example, Calabi-Yaus can be cut-out from Fano varieties.
	\item Fano varieties are special in that there are only finitely many of them in any given dimension.
\end{itemize}
\begin{thm}[Koll\'ar-Miyaoka-Mori]
	Up to \emph{deformation}, there are finitely many Fano varieties in each dimension.
\end{thm}
Here $X_1$ and $X_2$ are considered equivalent up to deformation if there exists a flat family $\mathcal{X}\to B$ over an irreducible base $B$ such that $X_1$ and $X_2$ are fibers over some points $b_1, b_2 \in B$. \vspace{1em}

This raises the big question: Can we classify the Fano varieties? The current progress is:
\begin{itemize}
	\item In dimension 1, there is just one Fano; $\bP^1$.
	\item In dimension 2, there are 10, called the del Pezzo surfaces.
	\item In dimension 3, there are 105, which were classified throughout the 70s and 90s.
	\item All higher dimensions are yet to be classified.
\end{itemize}
In this course, we are also concerned with \emph{mirror symmetries} for Fano varieties. A conjectured mirror symmetry is between $n$-dimensional Fano varieties and Laurent polynomials in $n$-dimensions up to an equivalence called \emph{mutation}. Loosely, we can say
\begin{defn}
	A variety $X$ is \emph{mirror} to a polynomial $f$, if you can determine \emph{enumerative info} about $X$ from $f$.
\end{defn} 
By enumerative info for $X$, we mean things like Gromov-Witten invariants, quantum cohomology and quantum periods. The mirror symmetry conjecture is that these can be computed in terms of correponding quantities of $f$. For example:
\begin{defn}
	Let $f \in \C[x_1^{\pm 1},...,x_n^{\pm 1}]$ be a Laurent polynomial. The \emph{classical period} of $f$ is the quantity
	\begin{align*}
		\pi_f(t) &= \int\limits_{(S^1)^n} \frac{1}{1-tf} ~\frac{dx_1}{x_1}...\frac{dx_n}{x_n}\\
		&= \sum_{k=0}^\infty \frac{c_1(f^k)}{k!}t^k.
	\end{align*}
	where $c_1(f^k)$ means the coefficient of the constant term of $f^k$.
\end{defn}
The idea therefore, is that mirror symmetry can help us compute hard things in geometry by using easier polynomial data. Then what is the status of this conjecture?
\begin{itemize}
	\item Established in dimensions 1 and 2 by checking all cases.
	\item In dimension 3, all Fanos $X$ have a mirror polynomial $f$, but the classification of other maximally mutable polynomials $f$ is unknown.
	\item In all dimensions, the symmetry is established for \emph{toric varieties}.
\end{itemize}
Toric varieties are Fano varieties which have the form $V\sslash T$ where $V$ is a (complex) vector space and $T=(\C^\ast)^k$, which is called the \emph{algebraic torus}. The double slash $\sslash$ indicates a \emph{geometric invariant theory} quotient, which will be discussed in the first part of the course. Recently, there is a lot of work on extending mirror symmetry to GIT quotients $V\sslash G$ more generally, for $G$ a reductive algebraic group. \vspace{1em}

As we will see, toric varieties are very nice to work with. This is because they are extremely computable. Essentially, there is a dictionary between the geometry of a toric variety $X$ and the combintorics of a polytope $P$ corresponding to $X$. The basic question then, is can a similar correspondence be generalised to other Fano varieties? What should play the role of the polytope? Mirror symmetry answers this question: $X$ corresponds to $f$, which has a \emph{Newton polytope} with some additional coefficient data. \vspace{1em}

Exercises:
\begin{enumerate}
	\item Show that a degree $d$ hypersurface in $\bP^n$ is Fano for $d<n+1$.
	\item Find a closed formula for the classical period of $f(x,y)=x+y+\frac{1}{xy}$, and find a differential equation that it satisfies.
\end{enumerate}
\pagebreak

\section{Quotients in Algebraic Geometry}
\begin{defn}
	An \emph{algebraic} group is a group which is also an algebraic variety. An \emph{action} of an algebraic group $G$ on a variety $X$ is a morphism 
	\begin{align*}
		G\times X &\to X\\
		(g, x) &\to g\cdot x
	\end{align*}
	such that for all $g,g' \in G$ and $x\in X$, we have $(gg')\cdot x = g\cdot (g'\cdot x)$ and $e\cdot x = x$.
\end{defn}
For example: $\C^\ast$, $GL(n)$ and $SL(n)$. \vspace{1em}

\begin{defn}
	Given an action of $G$ on $X$ and some $x\in X$, the \emph{orbit} of $x$ is
	\begin{equation}
		G\cdot x = \{g\cdot x, ~|~g\in G\}.
	\end{equation}
	The \emph{stabiliser} of $x$ is 
	\begin{equation}
		G_x = \{g\in G \text{s.t.} g\cdot x = x\}.
	\end{equation}
	Note that $G_x$ is a closed subgroup of $G$. 
\end{defn}
Example: Let $T=\C^\ast$ and $V=\C^2$. Define an action of $T$ on $V$ by $\lambda\cdot(z_1,z_2) = (\lambda z_1, \lambda z_2)$ for all $\lambda \in T$, $(z_1,z_2)\in V$. Then the orbit of $(z_1,z_2)\neq(0,0)$ is the line through $(z_1,z_2)$, except the origin. The stabilizer is $\{1\}$. If $(z_1,z_2)=(0,0)$ then the orbit is $\{(0,0)\}$ and the stabilizer is all of $T$. Notice that $(0,0)$ is in the closure of $G\cdot z$ for all $z\in \C^2$. 
\begin{prop}
	For any $G$, $X$ and $x\in X$, 
	\begin{itemize}
		\item The orbit $G\cdot x$ is locally closed and a smooth subvariety of $X$.
		\item Each of its irreducible components has dimension $\dim(G)-\dim(G_x)$.
		\item The closure of $G\cdot x$ is a union of $G\cdot x$ and orbits of strictly smaller dimension.
	\end{itemize}
	The last point implies that minimal dimension orbits must be closed, and $\overline{G\cdot x}$ always contains a closed orbit.
\end{prop}
\begin{defn}
	The action of a group $G$ is called \emph{closed} if every orbit of $G$ is closed.
\end{defn}
\begin{defn}
	A \emph{linear algebraic group} is a closed subgroup of $GL(n)$.
\end{defn}
Goal: If we have an action $G\circlearrowright X$, we want to build some quotient $X/G$ in an algebraio-geometric way. As a naive attempt we can just take the quotient as topological spaces. Consider the action of $\C^\ast$ on $\C^2$ from before. If we endow the set $\C^2/\C^\ast$ with the quotient topology, then since $[(0,0)]$ is in every open neighbourhood of every other point (as we can always take a sequence of $\lambda_i\in\C^\ast$ approaching zero), this quotient is not even Hausdorff. \vspace{1em}

To solve this, we essentially want to delete the origin, and obtain $\left(\C^2 - \{(0,0)\}\right)/\C^\ast = \bP^1$. The putative quotient $Y$ we want to define must have the following properties:
\begin{itemize}
	\item There exists a surjection $p:X\to Y$ which is $G$-invariant.
	\item $Y$ is separated.
	\item $Y$ satisfies the following universal property: if $f:X\to Z$ is $G$ invariant, then it factors uniquely through $p$. That is:
	\[\begin{tikzcd}
		X & Y \\
		& Z \\
		\\
		\\
		& {}
		\arrow["p", from=1-1, to=1-2]
		\arrow["f"', from=1-1, to=2-2]
		\arrow[dashed, from=1-2, to=2-2]
	\end{tikzcd}\] 
	\item For all $U$ open, $\OO_Y(U) \cong \OO_X(p^{-1}(U))^G$, where the superscript denotes $G$-invariant functions.
	\item If $Z\subset X$ is closed and $G$-invariant, then $p(Z)$ is closed. If $Z_1, Z_2$ are disjoint and closed then $p(Z_1)$ and $p(Z_2)$ are disjoint.
\end{itemize}
If $p$ satisfies all these properties, we say it is a \emph{good quotient}. Moving forward, we will talk about affine and projective GIT quotients, symplectic reduction, and comparison between these two methods of constructing good quotients. \vspace{1em}

We can also consider a geometric quotient, which is a good quotient whose points are orbits of $G\circlearrowright X$. 

Remark: The properties of being good or geometric are local on the base, meaning that $p:X\to Y$ is good or geometric if and only if there exists an open cover of $Y$ with the restrictions of $p$ being good or geometric.

\begin{lemma}
	If $p:X\to Y$ is good, then it is categorical.
\end{lemma}
\begin{proof}
	Suppose $g:X\to Z$ is another $H$ invariant morphism and $p:X\to Y$ is good. Then we want to define $h:Y\to Z$ such that $p\circ h = g$. Consider $g(p^{-1}(y))$ for some $y\in Y$, which we claim is a singleton set. Suppose for contradiction that there are $z_1\neq z_2 \in g(p^{-1}(y))$. Then $g^{-1}(z_1)\cap g^{-1}(z_2) = \emptyset$, and these are closed, G-invariant sets because $g$ is continuous and $G$-invariant. Hence, by the hypothesis that $p$ is good we have:
	\begin{equation}
		p(g^{-1}(z_1))\cap p(g^{-1}(z_2)) = \emptyset.
	\end{equation}
	However, we must also have that $y\in p(g^{-1}(z_i)), i=1,2$ because $z_i \in g(p^{-1}(y))$; hence we have a contradiction and must have that $g(p^{-1}(y))$ is a singleton. \vspace{1em}
	
	Therefore, we can define a map $h:Y\to Z$ by $y \to g(p^{-1}(y))$ and it is well-defined and clearly $p\circ h = g$. It remains to show that this is a morphism of schemes (namely we need it to be locally induced by ring morphisms $\OO_X \to \OO_Y$). Let $\{U_i\}$ be a finite open affine cover of $Z$. Let $W_i = X - g^{-1}(U_i) = g(U_i)^c$. Then $U_i$ being open implies that $g^{-1}(U_i)$ is open and hence $W_i$ is closed. Similarly, $g$ being $G$-invariant implies $W_i$ is also. Finally, since $U_i$ is a cover, $\bigcap W_i = \emptyset$. Thus by goodness of $p$ we have that $p(W_i)$ are all closed and 
	\begin{equation}
		\label{e:Wi-disjoint}
		\bigcap p(W_i) = \emptyset.
	\end{equation}
	Define $V_i = Y-p(W_i)=p(W_i)^c$ Then the $V_i$ are an open cover of $Y$ by equation \ref{e:Wi-disjoint}. Note further that $p^{-1}(V_i)\subset g^{-1}(U_i)$. Thus we have a sequence of maps
	\begin{equation}
		\OO_Z(U_i) \xrightarrow \OO_X(g^{-1}(U_i))^G \xrightarrow{\text{res}|_{p^{-1}(V_i)}} \OO_X(p^{-1}(V_i))^G \cong_{p \text{ good}} \OO_Y(V_i).
	\end{equation}
	The $G$-invariance on the second ring comes from the invariance of $g$. The last isomorphism is one of the hypothesis conditions of $p$ being good. Thus since $U_i$ and $V_i$ are affine, this defines a local morphism $h_i:V_i\to U_i$, and it suffices to verify that $h|_{V_i} = h$. 
\end{proof}
\begin{prop}
	Let $p:X\to Y$ be a good quotient. Then 
	\begin{enumerate}
		\item $\overline{G\cdot x_1} \cap \overline{G\cdot x_2} \neq \emptyset \iff  p(x_1)=p(x_2)$.
		\item For alL $y\in Y$, there exists a unique closed orbit in $p^{-1}(y)$.
	\end{enumerate}
\end{prop}
\begin{proof}
	1) Suppose $\overline{G\cdot x_1} \cap \overline{G\cdot x_2} \neq \emptyset$. Since $p$ is continuous and constant on orbits, it is constant on orbit closures and hence $p(\overline{G\cdot x_1}) = p(\overline{G\cdot x_2})$ and in particular $p(x_1)=p(x_2)$. On the other hand if $\overline{G\cdot x_1} \cap \overline{G\cdot x_2} = \emptyset$, then since $p$ is good, the image of each orbit is disjoint. \vspace{1em}
	
	2) Suppose there exist two closed orbits in $p^{-1}(y)$. If they are not equal, then again since $p$ is continuous and constant on orbits they must be disjoint and hence their image is disjoint by goodness. However they must each contain $y$ so this is a contradiction. (Existence was not given in class, maybe it is obvious?)
\end{proof}
Now, let us construct good quotients for affine varieties.
\subsection{Affine GIT Quotient}
Suppose we have the action of a group $G$ on an affine variety $X$. Then $X=\Spec(\OO_X)$ by definition, and we want our quotient to be good; so we want it to have functions $\OO_X^G$. Thus the idea is to let our quotient be $\Spec(\OO_X^G)$. However $\OO_X^G$ is not finitely generated in general, so we will restrict ourselves to \emph{reductive} groups.
\begin{defn}
	Let $G$ be a linear algebraic group.
	\begin{itemize}
		\item We say $G$ is \emph{reductive} if every smooth connected unipotent normal subgroup of $G$ is trivial.
		\item We say $G$ is \emph{linearly reductive} if, for every representation $G\to GL(V)$ and every non-zero fixed point $v\in V$, there exists a homogeneous $G$-invariant degree-1 polynomial $f$ on $V$ such that $f(V)\neq 0$.
		\item We say $G$ is \emph{geometrically reductive} if, for every representation $G\to GL(V)$ and every non-zero fixed point $v\in V$, there exists a homogeneous $G$-invariant polynomial $f$ on $V$ such that $f(V) \neq 0$.
	\end{itemize}
\end{defn}
\begin{thm}
	The three properties above are equivalent over $\C$. 
\end{thm}
For example, $GL(n,\C)$, $SL(n,\C)$ and $PGL(n,\C)$ are reductive.
\begin{thm}[Nagata's Theorem]
	Let $G$ be geometrically reductive acting on a finitely generated $\C$-algebra $R$. Then $R^G$ is finitely generated. 
\end{thm}
\begin{lemma}
	Let $G$ be a geometrically reductive group acting on an affine variety $X$. Let $Z_1$ and $Z_2$ be two closed, $G$-invariant disjoint subsets of $X$. Then there exists a $G$-invariant function $\psi\in\OO_X^G$ such that $\psi(Z_1)=1$ and $\psi(Z_2)=0$.
\end{lemma}
\begin{proof}
	Firstly
	\begin{equation}
		\langle 1 \rangle = I(\emptyset) = I(Z_1\cap Z_2) = I(Z_1)+I(Z_2),
\end{equation}
therefore $1 = f_1 + f_2$ for some $f_1,f_2$ with $f_i(Z_j)=\delta_{ij}$. Claim: (c.f. Hoskins) The subspace spanned by $\{g\cdot f, ~|~ g\in G\} \subset \OO_X$ is $G$-invariant and finite dimensional. Therefore we can pick a basis $h_1,..,h_n$, and because $G$ acts on all of the $h_i$, we get an induced action of $G$ on $\C^n$ such that the map
\begin{align*}
	\phi:X &\to \C^n\\
	x&\to (h_i(x))
\end{align*}
is $G$-equivariant, meaning $\phi(g\cdot x) = g\phi(x)$.  Note then that $\phi(Z_1)=0$ and $\Phi(Z_2)\neq 0$, and define $v=\Phi(Z_2) \in \C^n$. \vspace{1em}

Since $\phi$ is $G$-equivariant, $v$ is fixed by the action of $G$. Then by the hypothesis of geometric reductivity, there exists some $G$-invariant homogenous $f_0$ such that $f_0(v)\neq 0$ and $f_0(0)=0$. Finally, let 
\begin{equation}
	\psi = \frac{1}{f_0(v)}f_0\circ \phi.
\end{equation}
\end{proof}

\begin{defn}
	The \emph{affine GIT quotient} of an affine variety $X$ under reductive group $G$, denoted $X\sslash G$ is $\Spec(\OO_X^G)$.
\end{defn}
\begin{thm}
	Let $X$ be an affine variety and $G$ a reductive group acting on $X$. Then $p:X\to Y=\Spec(\OO_X^G)$ is a good quotient.
\end{thm}
\begin{proof}
	First we show $p$ is $G$-invariant. Suppose for contradiction there exist $x\in X$, $g\in G$ such that $p(x) \neq p(g\cdot x)$. Since $Y$ is affine, there exists an $x\in \OO_Y$ such that $f(x)\neq f(g\cdot x)$. However $\OO_Y = \OO_X^G$ by definition, so $f$ must be $G$ invariant, giving a contradiction. \vspace{1em}
	
	Next we show $p$ is surjective. Let $y\in Y$ and let $\langle f_1,...,f_n\rangle$ be the ideal defining $y$. Let $\m$ be the maximal ideal containing $\langle f_1,...,f_n\rangle$. The point corresponding to $\m$ in $X$ maps to $y$ under $p$. \vspace{1em}
	
	Now let $U\subset Y$ be open. We want to show $\OO_Y(U)\cong \OO_X(p^{-1}(U))^G$; it suffices to show this for $U=D_f^Y$ for any $f\in \OO_Y$.
	\begin{align*}
		\OO_Y(D_f^Y) &= (\OO_Y)_f\\
		&= [\OO_X(X)^G]_F\\
		& =[\OO_X(X)_f]^G\\
		&=[\OO_X(D_f^X)]^G\\
		&=\OO_X(p^{-1}(D_f^Y))^G
	\end{align*} 
	Let $Z_1,Z_2$ be $G$-invariant closed disjoint subsets. By the lemma, there exists $\psi \in \OO_X^G$ with $\psi(Z_1)=0$ and $\psi(Z_2)=1$. Then $\overline{p(Z_1)}\cap\overline{p(Z_2)}=\emptyset$, because there is a $G$-invariant function which separates them. This turns out to be equivalent to the topological condition for goodness, that $p(Z_1)\cap p(Z_2)=\emptyset$. To prove this, it suffices to prove that if $Z$ is closed and $G$-invariant then $p(Z)$ is closed. \vspace{1em}
	
	Suppose $Z$ is closed and $G$-invariant. For contradiction, suppose there exists $g\in \overline{p(Z)}-p(Z)$. Then $Z$ and $p^{-1}(y)$ are both closed and $G$-invariant, so 
	\begin{equation}
		\overline{p(Z)}\cap\overline{p(p^{-1}(y)} = \emptyset.
	\end{equation}
however $y$ must be in this intersection, giving a contradiction.
\end{proof}
\begin{prop}
	If the action of $G$ is closed then $X\sslash G$ is a geometric quotient.
\end{prop}
This GIT construction separates orbits as much as possible while still being good.

Recall the example of $\C^\ast \circlearrowright \C^2$ by scaling. We saw that $(0,0)$ is in the closure of every orbit. Hence $\C^2/\C^\ast$ is just one point. This is the same as saying that all the $\C^\ast$ invariants in $\C[x_1,x_2]$ are just the constants. Projective GIT will allow us to loosen the definition of $G$-invariance and get that $\C^2\sslash \C^\ast = \P^1$. Recall, if $f$ is homogeneous of degree $k$ then $f(\lambda x, \lambda y)= \lambda^k f(x,y)$, so $f$ is projectively invariant.


\end{document}
